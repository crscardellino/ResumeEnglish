%%%%%%%%%%%%%%%%%%%%%%%%%%%%%%%%%%%%%%%
% This is a modified ONE COLUMN version of
% the following template:
% 
% Deedy - One Page Two Column Resume
% LaTeX Template
% Version 1.1 (30/4/2014)
%
% Original author:
% Debarghya Das (http://debarghyadas.com)
%
% Original repository:
% https://github.com/deedydas/Deedy-Resume
%
% IMPORTANT: THIS TEMPLATE NEEDS TO BE COMPILED WITH XeLaTeX
%
% This template uses several fonts not included with Windows/Linux by
% default. If you get compilation errors saying a font is missing, find the line
% on which the font is used and either change it to a font included with your
% operating system or comment the line out to use the default font.
% 
%%%%%%%%%%%%%%%%%%%%%%%%%%%%%%%%%%%%%%

\documentclass[]{deedy-resume-openfont}


\begin{document}

%%%%%%%%%%%%%%%%%%%%%%%%%%%%%%%%%%%%%%
%
%     TITLE NAME
%
%%%%%%%%%%%%%%%%%%%%%%%%%%%%%%%%%%%%%%

\namesection{Cristian}{Cardellino}{
\urlstyle{same}\url{https://crscardellino.github.io} \\
\href{mailto:cristian.cardellino@mercadolibre.com}{\texttt{cristian.cardellino@mercadolibre.com}} | +54
351 7898 120 }

\vspace{2em}

%%%%%%%%%%%%%%%%%%%%%%%%%%%%%%%%%%%%%%
%     EDUCATION
%%%%%%%%%%%%%%%%%%%%%%%%%%%%%%%%%%%%%%

\section{Education}

\runsubsection{Universidad Nacional de C\'ordoba}
\descript{| PhD. Computer Sciences}
\location{Facultad de Matem\'atica, Astronom\'ia, F\'isica y Computaci\'on | 
2013 -- 2018 | C\'ordoba, Argentina}
Thesis: ``{\it A Study on Semi-Supervised Methods for Spanish Verb Sense 
Disambiguation}''\\
Advisor: Prof. PhD. Laura Alonso i Alemany
\sectionsep

\runsubsection{Universidad Nacional de C\'ordoba}
\descript{| MSc. Computer Sciences}
\location{Facultad de Matem\'atica, Astronom\'ia, F\'isica y Computaci\'on | 
2008 -- 2013 | C\'ordoba, Argentina}
Minor: Natural Language Processing -- Computational Linguistics \\
GPA: 8.74 / 10.00
\sectionsep

% \runsubsection{Universidad Nacional de C\'ordoba}
% \descript{| BS. Computer Sciences}
% \location{Facultad de Matem\'atica, Astronom\'ia, F\'isica y Computaci\'on | 
% 2008 -- 2011 | C\'ordoba, Argentina}
% GPA: 8.74 / 10.00
% \sectionsep

%%%%%%%%%%%%%%%%%%%%%%%%%%%%%%%%%%%%%%
%     INDUSTRY EXPERIENCE
%%%%%%%%%%%%%%%%%%%%%%%%%%%%%%%%%%%%%%

\section{Industry}

\runsubsection{Mercado Libre}
\descript{| Research Scientist}
\location{2020 – Present | C\'ordoba, Argentina}
\begin{tightemize}
\item Working on research for multi-task item representation using
semi-supervised and self-supervised machine learning.
\end{tightemize}
\sectionsep

\runsubsection{Tappedout.net}
\descript{| Freelance Machine Learning Researcher – Software Developer}
\location{2018 – 2020 | C\'ordoba, Argentina}
\begin{tightemize}
\item Developed several recommender systems using collaborative filtering, matrix
factorization, content based filtering and hybrid approaches.
\item Implemented bots to make game decisions using neural networks with 
TensorFlow.
\item Implemented the use of Apache Airflow for task automation regarding data pipelines.
\item Created frontend applications with React.js and
backend APIs with Django/Flask as interface for the models.
\item Worked on log analysis for automatic detection of bots and spam using
clustering technologies with PySpark.
\end{tightemize}
\sectionsep

\runsubsection{Santex Am\'erica}
\descript{| Coordinator of Santex-FAMAF agreement}
\location{2018 – 2019 | C\'ordoba, Argentina}
\begin{tightemize}
\item Managed the agreement for developing the research on Artificial Intelligence applications in the industry.
\item In charge of in-company training and discussion of how and where to apply different artificial 
      intelligence tools for industry processes.
\end{tightemize}
\sectionsep

\runsubsection{INRIA}
\descript{| Research/Software Development Internship }
\location{2014 | Sophia Antipolis, France}
\begin{tightemize}
\item Developed the web application Licentia (\texttt{http://licentia.inria.fr}).
\item In charge of web design and user experience.
\end{tightemize}
\sectionsep

\runsubsection{Machinalis}
\descript{| Jr. Python/Django Internship}
\location{2012 | C\'ordoba, Argentina}
\begin{tightemize}
\item Developed a website for e-commerce in the construction industry.
\end{tightemize}
\sectionsep

%%%%%%%%%%%%%%%%%%%%%%%%%%%%%%%%%%%%%%
%     RESEARCH
%%%%%%%%%%%%%%%%%%%%%%%%%%%%%%%%%%%%%%

\section{Research \& Teaching}

\runsubsection{Data Science and Machine Learning Specialization -- UNC}
\descript{| Professor}
\location{Facultad de Matem\'atica, Astronom\'ia, F\'isica y Computaci\'on | 
2018 -- Present | C\'ordoba, Argentina}
These courses are designed to provide students with the knowledge and tools 
to understand and apply efficient statistical and machine learning techniques
for transformation and analysis of data. I've been so far in charge of designing
course material, both theory and practice, for the following courses: 
``introduction to machine learning'', ``supervised machine learning'', 
``introduction to deep learning'', and ``recommender systems''.
\sectionsep

\newpage

\runsubsection{Universidad Nacional de C\'ordoba}
\descript{| Professor}
\location{Facultad de Matem\'atica, Astronom\'ia, F\'isica y Computaci\'on | 
2015 -- Present | C\'ordoba, Argentina}
Professor for different courses where I design and grade deliverables to the
students using different technologies in the area. The courses I have been
are: ``natural language processing'', ``programming paradigms'', ``databases'', 
and ``operating systems''. I have also directed 3 Msc. Students in NLP related topics.
% \begin{tightemize}
% \item {\it Programming Paradigms}: Grade deliverable projects in different 
% programming languages with the use of different paradigms (imperative, 
% functional, object oriented).
% \item {\bf Databases}: Grade deliverable projects for creation and querying of 
% relational databases using SQL and non-relational databases using MongoDB.
% \item {\bf Operating Systems}: Create and grade deliverable projects for
% different elements of OSs (scheduler, file systems, concurrency, etc.).
% \item {\bf Software Engineering}: Guide and grade students during different
%   phases of a software project (requirements, architecture, coding, testing,
%   etc.)
% \end{tightemize}
\sectionsep

\runsubsection{Data Science and Machine Learning Tutorials}
\descript{| Instructor}
\location{2017 -- Present | C\'ordoba, Argentina}
I give different talks and tutorials in different areas of machine learning,
and data science. I also write some tutorials via blogging platforms like ``Medium''.
\begin{tightemize}
\item ``Introducci\'on a Redes Neuronales sobre Grafos''. Meetup Data Science Córdoba. Argentina. 2019.
% \item ``Introducci\'on a Apache Spark y Apache Zeppelin''. Facultad de Ciencias Qu\'imicas.
% Universidad Nacional de C\'ordoba. Argentina. 2019.
\item ``Procesando Datos con Spark''.
\href{https://medium.com/@crscardellino/procesando-datos-con-spark-48539d38e437}{Tutorial @ Medium.com}. 2019.
\item ``Train and visualize a model in TensorFlow''. PyData. Universidad 
Nacional de San Luis. Argentina. 2017.
% \item ``Express deep learning with Python''. Escuela Argentina de 
% Inteligencia Artificial. Universidad Tecnol\'ogica Nacional.
% C\'ordoba, Argentina. 2017.
\end{tightemize}
\sectionsep

\runsubsection{MIREL}
\descript{| PhD Researcher}
\location{2016 -- Present | C\'ordoba, Argentina -- Sophia Antipolis, France}
European project funded by the European Union's Horizon 2020 research and 
innovation programme under the Marie Sk\l{}odowska-Curie grant agreement No 
690974. Worked in named entity recognition for legal texts and argumentation
mining.
\begin{tightemize}
\item ``Convolutional ladder networks for legal NERC and the impact of unsupervised data
in better generalizations''. Cardellino, C., Alonso Alemany, L., Teruel, M., Villata, S.,
and Marro, S.. The 32nd International FLAIRS Conference, 2019, Sarasota, U.S.A.
\item ``Increasing Argument Annotation Reproducibility by Using Inter-annotator
Agreement to Improve Guidelines''. Teruel, M., Cardellino, C., 
Cardellino, F., Alonso Alemany, L., and Villata, S. The 11th International 
Conference on Language Resources and Evaluation (LREC), 2018, Miyazaki, Japan.
\item ``Ontology population and alignment for the legal domain: YAGO, Wikipedia and LKIF''
Cardellino, C., Teruel, M., Alemany, L.A., Villata, S. The 16th International
Semantic Web Conference (ISWC), 2017, Vienna, Austria.
\item ``A low-cost, high-coverage legal named entity recognizer, classifier and linker''
Cardellino, C., Alonso Alemany, L., Teruel, M., Villata, S. The 16th International 
Conference on Artificial Intelligence and Law (ICAIL), 2017, London, U.K.
\item ``Legal NERC with ontologies, Wikipedia and curriculum learning''. 2017.
Cardellino, C., Alonso Alemany, L., Teruel, M., Villata, S. The 15th Conference of the 
European Chapter of the Association for Computational Linguistics (EACL), 2017,
Valencia, Spain.
\end{tightemize}
\sectionsep

\runsubsection{CONICET}
\descript{| PhD Student Researcher}
\location{Mar 2013 – Apr 2018 | C\'ordoba, Argentina}
Scholarship to research in semi-supervised machine learning techniques applied
to natural language processing tasks.
\begin{tightemize}
\item ``Exploring the impact of word embeddings for disjoint semisupervised Spanish verb 
sense disambiguation''. 2018. Cristian Cardellino and Laura Alonso Alemany.
Inteligencia Artificial, [S.l.], v. 21, n. 61, p. 67-81, mar. 2018. ISSN 1988-3064.
\end{tightemize}
\sectionsep

\runsubsection{Spanish Billion Words Corpus and Embeddings}
\descript{| Creator and Maintainer}
\location{Mar 2016 – Present | C\'ordoba, Argentina}
Created and maintaining the Spanish Billion Word Corpus and Embeddings. A corpus 
of 1.5 billion words for the Spanish language along with embeddings trained with
word2vec. This project have given me a lot of insight in the ETL process.
\sectionsep

\runsubsection{Events}
\descript{| Organization}
\location{2018 -- 2019 | C\'ordoba, Argentina}
Part of the organizing committee of PyData C\'ordoba 2018 and 2019.
\sectionsep

% \runsubsection{Tutorials}
% \descript{| Instructor}
% \location{2017 | C\'ordoba, Argentina}
% % I have created and presented different courses and tutorials in different 
% % fields of Machine Learning, Data Science and Machine Learning:
% \begin{tightemize}
% % \item ``Combining Python Notebooks with R and ggplot''. $4^{\circ}$ 
% % Encuentro Data Science C\'ordoba 2017. Cluster C\'ordoba Technology. C\'ordoba, 
% % Argentina. Dec 2017.
% \item ``Train and visualize a model in TensorFlow''. PyData. Universidad 
% Nacional de San Luis. San Luis, Argentina. Nov 2017.
% \item ``Express deep learning with Python''. $1^{\circ}$ Escuela Argentina de 
% Inteligencia Artificial. Universidad Tecnol\'ogica Nacional. C\'ordoba. 
% C\'ordoba, Argentina. Sep 2017.
% \end{tightemize}
% \sectionsep

%%%%%%%%%%%%%%%%%%%%%%%%%%%%%%%%%%%%%%
%     COURSEWORK
%%%%%%%%%%%%%%%%%%%%%%%%%%%%%%%%%%%%%%

% \section{COURSEWORK}

% \runsubsection{Embeddings and Deep Learning}
% \descript{| ESSLLI}
% \location{Jul 2017 | European Summer School in Logic, Language and Information - 
% Toulouse, France.}
% \sectionsep

% \runsubsection{Distributed Computing and Data Analysis with Spark}
% \descript{| FAMAFyC}
% \location{Aug 2016 - Nov 2016 | Facultad de Matem\'atica, Astronom\'ia, 
% F\'isica y Computaci\'on. UNC. C\'ordoba, Argentina.}
% \sectionsep

% \runsubsection{Parallel Computing}
% \descript{| FAMAFyC}
% \location{Mar 2014 - Jun 2014 | Facultad de Matem\'atica, Astronom\'ia, 
% F\'isica y Computaci\'on. UNC. C\'ordoba, Argentina.}
% \sectionsep

% \runsubsection{Vectorial and neural models of compositional semantics}
% \descript{| ESSLLI}
% \location{Aug 2015 | European Summer School in Logic, Language and Information - 
% Barcelona, Spain.}
% \sectionsep

% \runsubsection{Machine Learning}
% \descript{| Coursera}
% \location{2014 | Stanford University by Coursera.}
% \sectionsep

% \runsubsection{Natural Language Processing}
% \descript{| FAMAFyC}
% \location{Aug 2013 - Nov 2013 | Facultad de Matem\'atica, Astronom\'ia, 
% F\'isica y Computaci\'on. UNC. C\'ordoba, Argentina.}
% \sectionsep

%%%%%%%%%%%%%%%%%%%%%%%%%%%%%%%%%%%%%%
%     TECHNICAL SKILLS
%%%%%%%%%%%%%%%%%%%%%%%%%%%%%%%%%%%%%%

\section{Technical Skills}

\runsubsection{Programming Languages}
\descript{}
\begin{tightemize}
\item Preferred: Python \textbullet{} Scala \textbullet{} JavaScript
\item Known: C/C++ \textbullet{} Ruby \textbullet{} 
BASH \textbullet{} R \textbullet{} SQL \textbullet{} Perl % \textbullet{} XML
\item Basic knowledge: PHP \textbullet{} Java % \textbullet{} CSS \textbullet{} 
% Octave \textbullet{} OCaml \textbullet{} Haskell \textbullet{} CUDA
\end{tightemize}
\sectionsep

\runsubsection{Frameworks and Technologies}
\descript{}
\begin{tightemize}
\item TensorFlow \textbullet{} Scikit-Learn \textbullet{} NLTK \textbullet{} Pandas \textbullet{} Jupyter 
\textbullet{} matplotlib \textbullet{} Spacy \textbullet{} Spark \textbullet{} R tidyverse
\textbullet{} Freeling
\item React.js \textbullet{} Django \textbullet{} jQuery 
\textbullet{} Flask % \textbullet{} Play Framework \textbullet{} Ruby on Rails
\item MySQL \textbullet{} PostgreSQL \textbullet{} MongoDB % \textbullet{} Apache Jena Fuseki
\item NGINX \textbullet{} Vagrant \textbullet{} Docker \textbullet{} Git \textbullet{} Apache Airflow
\end{tightemize}
\sectionsep

\subsection{Languages}
Native/bilingual fluency: Spanish -- English \textbullet{} Basic reading: 
Italian -- French
\end{document}
